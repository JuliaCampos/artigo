\section{\esp Metodologias utilizadas }\label{metodologia}

Com o objetivo de encontrar boas soluções para o \ac{PERE}, foram utilizadas diversas técnicas, como \acl{PLI}, Programação por Restrições, Enxame de Partículas e Algoritmos Genéticos, além de meta heurísticas.
Entre as técnicas mais utilizadas, destaca-se a Programação Linear Inteira, encontrado em 57,9\% dos estudos realizados, sendo seguido pela Programação por Restrições, além de algumas heurísticas e meta heurísticas baseadas em Algoritmos Genéticos e Enxame de Partículas, como mostrado na Tabela~\ref{tecnicas}. 

\begin{table}[ht]
\centering
\caption{Métodos identificados}
\label{tecnicas}
\begin{tabular}{c|c}
\hline
\textbf{Referências}                                                                                                                                                                                                                                                                                                                                                                                                                                                                                                                                                                                          & \textbf{Técnicas utilizadas} \\ \hline
\begin{tabular}[c]{@{}c@{}}
\cite{tozlu:2016}      \\
\cite{goos:2015}        \\
\cite{cheng:98}          \\
\cite{Decerle:2016}       \\
\cite{calvo:2013}          \\
\cite{trabelsi:2012}      \\ 
\cite{bachouch:2010}       \\ 
\cite{tricoire:2016}        \\ 
\cite{Kergosien:2009}        \\
\cite{Bertels:2006}           \\
\cite{rasmussenm:2012}         \\
\cite{Bierwirth:2013}           \\ 
\cite{trautsamwieser:2014}       \\
\end{tabular} & Programação Linear Inteira   \\ \hline
\begin{tabular}[c]{@{}c@{}}
\cite{cattafi:2012} \\
\cite{urli:2014} \\
\cite{Bertels:2006}\\
\end{tabular}                                                                                                                                                                                                                                                                                                                                                                                                                                                       & Programação por Restrições   \\ \hline
\begin{tabular}[c]{@{}c@{}}
\cite{mutingi:2013} \\ 
\cite{drake:2007}
\end{tabular}                                                                                                                                                                                                                                                                                                                                                                                                                                                                                             & Enxame de partículas         \\ \hline
\begin{tabular}[c]{@{}c@{}}
\cite{luna:2013} \\ 
\cite{nguyen:2016}\end{tabular}                                                                                                                                                                                                                                                                                                                                                                                                                                                                                               & Algoritmos evolutivos        \\ \hline
\end{tabular}
\end{table}

A partir dos dados apresentados anteriormente, podemos concluir que a \ac{PLI} tem sido a técnica mais utilizada na obtenção de resultados para o \ac{PERE}, retornando bons resultados quando testada com pequenas e médias instâncias, já que a técnica de Programação por Restrições tem se mostrado como a segunda alternativa mais estudada, trazendo bons resultados para a solução de problemas de escalonamento e roteamento.
Bons resultados pra instâncias grandes foram encontrados por \citeonline{nguyen:2016}, com a utilização de Algoritmos Evolutivos.

Além das técnicas utilizadas separadamente, para a obtenção de melhores resultados, alguns pesquisadores, obtiveram bons resultados ao combinar as técnicas de Programação Linear Inteira com  Programação por Restrições, como elaborado por \cite{Bertels:2006}. 

De modo geral, os métodos exatos tem sido muito utilizados, o que deixa uma oportunidade para a utilização de heurísticas e meta heurísticas para solucionar o \ac{PERE} e para tratar de instâncias de grande porte.
