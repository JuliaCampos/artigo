\section{\esp Considerações finais }\label{conclusão}

O~\ac{SAD} é um projeto relevante socialmente, servindo como uma opção viável à internação hospitalar para pacientes em estado estável que ainda necessitam de cuidados médicos. Esse serviço possui vantagens, tais como a redução do risco de infecção hospitalar e maior conforto ao paciente. E desvantagens, como o aumento de custos ao gestor do domicílio e custos relacionados ao deslocamento dos veículos encarregados de transportar as equipes de atendimento.

Atualmente o \ac{PERE} tem sido solucionado de forma manual em diversos países. Levando em consideração a classe de complexidade deste problema, diversos pesquisadores vêm investigando técnicas de otimização combinatória para obter melhores soluções. Com o objetivo de identificar quais técnicas de solução foram aplicadas ao \ac{PERE} e formalizar e garantir a replicabilidade do estudo, foi elaborada uma Revisão Sistemática de Literatura, cuja questão de pesquisa foi respondida nas Seções \ref{pad} e \ref{metodologia} com a análise dos materiais bibliográficos obtidos.

Após a busca e seleção dos materiais retornados na pesquisa, é  possível observar que a maior parte dos artigos encontrados foram publicados entre os anos de 2012 e 2016, o que indica um aumento do interesse por parte dos pesquisadores em estudar este tema neste período de tempo. 

Entre as técnicas de solução investigadas, foi notado que os pesquisadores utilizaram diversas técnicas para elaborar soluções para o~\ac{PERE}, tendo os principais objetivos: A minimização da carga horária das enfermeiras e da rota das enfermeiras e a maximização de satisfação dos pacientes e enfermeiras. Também foram encontrados alguns estudos de caso, indicando a existência de uma parceria entre os pesquisadores e algumas empresas de Atenção Domiciliar ou de Hospitais para a solução de problemas específicos.

Assim, é possível inferir que dentre todas as técnicas utilizadas pelos pesquisadores, a técnica de Programação Inteira Linear tem se mostrado um dos métodos mais utilizados para solucionar o~\ac{PERE}, e que os estudos de Programação por Restrições também estão ganhando atenção da comunidade científica. 

Observou-se também que grande parte dos trabalhos utilizam instâncias de pequeno porte e o volume de publicações que apresentam técnicas baseadas em heurísticas e meta heurísticas ainda é pequeno. Com isso acredita-se que se faz necessário desenvolver novos modelos de Programação Linear Inteira e de Programação por Restrições, ou ainda, desenvolver heurísticas capazes de encontrar soluções de boa qualidade para instâncias de médio e grande porte.
