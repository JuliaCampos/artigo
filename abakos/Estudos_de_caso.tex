\section{Estudos de Casos\label{casos}}

Ao longo da condução da Revisão Sistemática de Literatura foram identificados alguns trabalhos com características únicas, pois tratam-se de estudos de casos reais, organizados na Tabela~\ref{estudo}, indicando a existência de parcerias entre os pesquisadores e instituições que prestam Serviço de Atenção Domiciliar.

\begin{table}[H]
\centering
\caption{País do Estudos de Caso}
\label{estudo}
\begin{tabular}{c|c}
\hline
\textbf{Autor}                  & \textbf{Estudo de Caso} \\ \hline
\cite{tozlu:2016}          		& Turquia    \\ \hline
\cite{cattafi:2012}        		& Itália      \\ \hline
\cite{rasmussenm:2012}		   	& Dinamarca	   \\ \hline
\cite{trautsamwieser:2014} 		& Áustria      \\ \hline
\cite{luna:2013}           		& Espanha      \\ \hline
\cite{goos:2015}           		& Bélgica        \\ \hline
\cite{drake:2007}		 	   	& Inglaterra	  \\ \hline	
\cite{nguyen:2016}         		& Suíça         \\ \hline
\end{tabular}
\end{table}

Algumas empresas como o Grupo Eulen, uma multinacional com sede na Espanha, o Grupo Landelijke Thuiszorg, com sede na Bélgica, e outras empresas na Suíça e na Turquia, realizaram um trabalho junto a \cite{luna:2013}, \cite{goos:2015}, \cite{nguyen:2016} e \cite{tozlu:2016}. 
Além das empresas privadas, alguns estudos de caso também foram realizados junto a hospitais ou a empresas pertencentes ao governo local, como em \cite{trautsamwieser:2014}, na Áustria, \cite{cattafi:2012} na Itália, \cite{drake:2007}, na Inglaterra e \cite{rasmussenm:2012}, na Dinamarca.

Os estudos de caso de \cite{luna:2013}, \cite{trautsamwieser:2014} e \cite{cattafi:2012}, tiveram como objetivo equilibrar a carga horária das enfermeiras, sendo que além deste objetivo, \cite{luna:2013} também objetivou minimizar o tamanho da equipe de atendimento.
Já \cite{goos:2015} e \cite{tozlu:2016} realizaram seu estudo de caso buscando minimizar a distância percorrida pelas enfermeiras. Além de minimizar a distância percorrida, \cite{goos:2015} também teve como objetivo maximizar a preferência dos clientes.

Foi introduzido por \cite{nguyen:2016} um modelo do \ac{PERE} cujas características como incerteza na disponibilidade da enfermeira, restrições legais de horário e de trabalho são levadas em consideração, tendo como objetivo solucionar problemas de: ~\textit{rostering, assignment, routing, and scheduling}.

Em seu estudo de cado, \cite{trautsamwieser:2014} selecionou as enfermeiras em níveis, sendo as enfermeiras de nível mais baixo encarregadas de prestar serviços de apoio ao paciente, como dar banho ou auxiliar nas tarefas domésticas e as enfermeiras de nível mais elevado, prestam serviços médicos e de enfermagem, como aplicar medicamentos.

%
% Comentei por que estava muito estranho
%
%enquanto \cite{tozlu:2016} classificou os pacientes entre os tipos 1, tipo 2 e tipo 3. Os pacientes de tipo 1 necessitam de enfermeiras, os pacientes de tipo 2 necessitam apenas de cuidadores e os pacientes de tipo 3 necessitam tanto de uma enfermeira quanto de cuidadores.

Em um estudo de caso na Itália \cite{cattafi:2012}, observou-se que ao solucionar os problemas de minimização de distância e equilíbrio de carga horária, consequentemente os problemas relacionados ao atraso de atendimento e alguns problemas de satisfação dos clientes, como ser atendido sempre pela mesma enfermeira, seriam solucionados.

Os experimentos foram realizados em \cite{nguyen:2016} com dados reais, compostos por 190 pacientes, 760 serviços, efetuados por 15 enfermeiras dentro de um período de sete dias, obtidos em uma empresa de Atenção Domiciliar em Lugano, Suíça.

Com o objetivo de minimizar os atrasos e cancelamentos, o estudo de caso de \cite{rasmussenm:2012} foi realizado junto ao governo da Dinamarca, onde as visitas possuem duração entre duas a quatro horas, as enfermeiras possuem diferentes carga horárias de trabalho e são responsáveis pelo meio de transporte com os quais irá atender os pacientes e as visitas possuem uma relação de precedência. Em \cite{drake:2007} apresentou-se um estudo de caso junto ao governo da Inglaterra, 
com o objetivo de otimizar o cronograma de atendimento, que era elaborado de forma manual pela equipe do \ac{PERE} do governo local.

Apesar de não se tratar de um estudo de case, o artigo de \cite{Bierwirth:2013} destaca-se pela diferenciação dos tipos de serviços entre simples e duplos. 
Um serviço simples consiste em um serviço a ser executado por um único membro da equipe, enquanto um serviço duplo consiste em pelo menos uma atividade ou operação de serviço que é realizada por dois membros da equipe. 
Os serviços duplos foram divididos em serviços simultâneos e serviços com uma determinada relação de precedência.  
Um serviço simultâneo ocorre quando existe a necessidade de mais de uma enfermeira para executar determinado serviço. Por exemplo, levantar uma pessoa com deficiência requer dois membros de uma equipe. 
Serviços com relação de precedência ocorrem quando a ordem de realização dos serviços é importante, por exemplo, a administração de medicamentos antes do fornecimento de uma refeição.
